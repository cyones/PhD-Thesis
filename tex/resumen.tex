\newpage
\thispagestyle{fancy}
\vspace{3cm}
\begin{center}
{\huge \textbf{Resumen}}
\end{center}
\vspace{1cm}

El aprendizaje maquinal ha tenido un gran desarrollo en los últimos años y ha permitido resolver una gran cantidad de problemas en las más diversas disciplinas.
Sin embargo, aún quedan grandes desafíos por resolver, como lo es el aprendizaje en datos con alto grado de desbalance de clases o con muy pocos datos
etiquetados. Un caso particular de aplicación donde se presentan desafios como estos es en la predicción computacional de secuencias de microARN (miARN).  Los
microARN (miARN) son un grupo de pequeñas secuencias de ácido ribonucleico (ARN) no codificante que desempeñan un papel muy importante en la regulación génica.
En los últimos años, se han desarrollado una gran cantidad de métodos que intentan detectar nuevos miARNs utilizando sólo información de estructura y secuencia,
es decir, sin medir niveles de expresión. El primer paso en estos métodos generalmente consiste en extraer del genoma subcadenas de nucleótidos que cumplan con
ciertos requerimientos estructurales. En segundo lugar se extraen características numéricas de estas subcadenas para finalmente usar aprendizaje maquinal para
predecir cuáles probablemente contengan miARN. Por otro lado, en paralelo con los métodos de predicción de miARN se han propuesto una gran cantidad de
características para representar numéricamente las subcadenas de ARN. Finalmente, la mayoría de los métodos actuales usan aprendizaje supervisado para la etapa
de predicción. Este tipo de métodos tienen importantes limitaciones prácticas cuando deben aplicarse a tareas de predicción real.  Existe el desafío de lidiar
con un número escaso de ejemplos de pre-miARN positivos. Además, es muy difícil construir un buen conjunto de ejemplos negativos para representar el espectro
completo de secuencias no miARN. Por otro lado, en cualquier genoma, existe un enorme desequilibrio de clase ($1:10000$) que es bien conocido por afectar
particularmente a los clasificadores supervisados.

Para permitir predicciones precisas y rápidas de nuevos miARNs en genomas completos, en esta tesis se realizaron aportas en las tres etapas del proceso de
predicción de miARN.  En primer lugar, se desarrolló una herramienta para extraer subcadenas de un genoma completo que cumplan con los requerimientos mínimos
para ser potenciales pre-miARNs miARN. En segundo lugar, se desarrolló una herramienta que permite calcular la mayoría de las características utilizadas para
predicciones de miARN en el estado del arte. La tercer y principal contribución consiste en un algoritmo novedoso de aprendizaje semi-supervisado que permite
realizar predicciones a partir de muy pocos ejemplos de clase positiva y el resto de las cadenas sin etiqueta de clase. Este tipo de aprendizaje aprovecha la
información provista por las subcadenas desconocidas (sobre las que se desea generar predicciones) para mejorar las tasas de predicción. Esta información extra
permite atenuar el efecto del número reducido de ejemplos etiquetados y la pobre representatividad de las clases. Cada herramienta diseñada fue comparada contra
el estado del arte, obteniendo mejores tasas de desempeño y menores tiempos de ejecución.

\newpage
\thispagestyle{empty}
