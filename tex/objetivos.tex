\section{Objetivo general}

El objetivo general de esta tesis es desarrollar nuevos métodos para la clasificación de candidatos a pre-miARN a partir del  genoma completo tanto de
animales como de plantas, y capaces de desempeñarse adecuadamente en un contexto de gran desbalance entre la clase de interés y una alta proporción de
secuencias no etiquetadas.

\section{Objetivos específicos}

A continuación se detallan los objetivos específicos de la presente investigación:
\begin{itemize}
	\item Desarrollar una metodología integrada y simple que permita encontrar automáticamente secuencias tipo tallo-horquilla en genomas completos.
	\item Revisar e integrar todas las características utilizadas en la actualidad para la predicción de pre-miARNs.
	\item Desarrollar nuevos algoritmos de clasificación que incorporen etapas de aprendizaje semi-supervisado para atacar los problemas de gran
		desbalance de clases y alta proporción de ejemplos no etiquetados.
	\item Validar las propuestas con genomas reales a través del trabajo multidisciplinario con expertos del dominio.
\end{itemize}

