\chapter{Conclusiones}
En esta tesis doctoral, presentamos una nueva metodología de predicción pre-miARN compuesta por tres etapas. Para la primer etapa se desarrolló un método para
extraer secuencias con estructura secundaría tipo horquilla del genoma completo. Este método contempla además el recorte de las secuencias y la eliminación de
horquillas repetidas, tanto idénticas como similares. Las pruebas en los genomas de 6 especies demostraron que el método propuesto encuentra una mayor cantidad
de horquillas que el único método disponible para esta tarea y, lo que es más importante aún, una mayor cantidad de pre-miARNs. El método con el cual se comparó
encontró en cada genoma aproximadamente un 90\% de los pre-miARNs conocidos llegando a un 80\% en el caso de una especie no modelo.  En cambio la metodología
propuesta alcanzó valores cercanos al 100\% en todas las especies.

En segundo lugar, se realizó una exhaustiva revisión del estado del arte para analizar que características se utilizan en al predicción de miARNs. Con esta
información se desarrollo una biblioteca de algoritmos de extracción de características y una versión web para utilizar estos algoritmos sin la necesidad de tener
conocimientos sobre programación. Consideramos además que la revisión del estado del arte y la posterior recopilación de información puede ser de gran valor
para la comunidad.

Por último, se desarrolló un algoritmo de predicción que utiliza un enfoque semi-supervisado para enfrentar el problema de ejemplos de entrenamiento escasos y
poco confiables. Los experimentos realizados en una configuración supervisada forzada mostraron que alcanza las tasas de clasificación de los mejores métodos
del estado del arte en pruebas de validación cruzada estándar y en tiempos más cortos. El método propuesto también se probó en condiciones que están más cerca
de una tarea de predicción real, donde se reduce el número de secuencias etiquetadas. En estas pruebas, superó claramente al mejor método supervisado
disponible, el cual sufrió una importante caída de desempeño al reducirse la cantidad de ejemplos positivos disponibles. Ademas se obtuvieron mejores resultados
que otro método diseñado especialmente para trabajar con pocos ejemplos de entrenamiento. Para la última prueba se procesaron tres genomas completos de especies
modelo y se compararon resultados con varios métodos del estado del arte. En esta prueba una gran cantidad de métodos fallaron porque no fueron capaces de
procesar la gran cantidad de secuencias de un genoma completo. Los métodos que funcionaron obtuvieron desempeños por debajo del obtenido con el método
desarrollado, en algunos casos cercanos al error de un clasificador al azar.


Los resultados a lo largo de este trabajo permiten arribar a las siguientes conclusiones:
\begin{itemize}
\item Existía una necesidad de mejorar el proceso de extracción de horquillas del genoma completo, dado que los métodos existentes no se desempeñaban
	satisfactoriamente y una gran cantidad de pre-miARNs se perdían en este temprano paso. La metodología diseñada logró superar esta falencia.
\item La biblioteca de extracción de características desarrollada permite calcular la gran mayoría de las características utilizadas en la predicción de miARNs
	en la actualidad.
\item El método de predicción desarrollado demostró ser escalable, a diferencia de los métodos del estado del arte que en su mayoría no lograron funcionar con
	genomas completos.
\item Los ejemplos negativos que se utilizan para entrenar muchos métodos de predicción del estado del arte no son representativos de la clase completa de
	secuencias no miARN.
\item Los métodos supervisados de predicción de miARN logran tasas de desempeño muy altas en pruebas de validación cruzada con una clase negativa definida
	artificialmente, pero el rendimiento disminuye en gran medida cuando tienen que enfrentar la diversidad de secuencias que se pueden encontrar en un
	genoma real. Por lo tanto, esta metodología de validación no permite obtener conclusiones confiables sobre el desempeño de las distintas técnicas de
	predicción.
\item El método de predicción desarrollado resulta eficiente y escalable, lo que permite procesar genomas completos con varios millones de horquillas.
\item El método de predicción desarrollado busca automáticamente una amplia variedad de ejemplos negativos entre las secuencias tipo horquilla. Ademas, al ser
	un algoritmo de aprendizaje semi-supervisado, tiene en cuenta la distribución de las secuencias sin etiqueta en el espacio de las características para
	ajustar fronteras de decisión alrededor de los pre-miARNs. Esto permite un mejor desempeño que el de los métodos del estado del arte.
\end{itemize}

